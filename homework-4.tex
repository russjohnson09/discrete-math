\documentclass[12pt]{article}
%\usepackage{geometry}
\usepackage[a4paper,centering,includehead]{geometry}
%\usepackage[paperwidth=6in, paperheight=9in]{geometry} %For Book Printing
\usepackage{changepage}
\usepackage{color}
\usepackage[usenames,dvipsnames,svgnames,table]{xcolor}
%==============================
\usepackage{microtype}
%==============================
%Math Related Packages
\usepackage{amsmath, amsfonts, amssymb, amsthm, xfrac, mathtools}

\newtheorem{thm}{Theorem}

\begin{document}
\begin{flushright}
Russ Johnson\\
Homework 4\\
\today\\
\end{flushright}

\noindent Consider the following:
\begin{quote}
Cameron has vowed to increase his vocabulary by learning the meanings of 90 new words over the course of 53 days learning at least one new word each day.
\end{quote}

\begin{thm}
Over some span of consecutive days, Cameron will learn exactly 15 new words.
\end{thm}

\begin{proof}
We will prove this theorem directly using the Pigeon-hole Principle. Let $a_i$ be the words learned up until and including the $i^{th}$ day. We know that
\begin{equation}\label{1}
1 \leq a_1
\end{equation}
because each day Cameron learns at least one word. From this same fact we know that
\begin{equation}\label{2}
a_j < a_{j+1}
\end{equation}
for all $j\in\{x\in \mathbb{Z}:0<x<53\}$.
Also since Cameron learns 90 words by the end,
\begin{equation}\label{3}
a_{53} = 90
\end{equation}
From \eqref{1}, \eqref{2}, and \eqref{3}, we can conclude
\begin{equation}\label{4}
1 \leq a_1 < a_2 < a_3 < \cdots < a_{53} \leq 90.
\end{equation}
Adding 15 to each part of \eqref{4}, we obtain
\begin{equation}\label{5}
16 \leq a_1+15 < a_2+15 < a_3+15 < \cdots < a_{53}+15 \leq 105.
\end{equation}
Let the set $A = \{a_1,a_2,\ldots,a_{53},a_1+15,a_2+15,\ldots,a_{53}+15\}$. Now we see that 
\begin{equation}\label{6}
|A| = 106
\end{equation}
We will now use the Pigeon-hole Principle for this last part. From \eqref{4} and \eqref{5} we see that we can use the numbers 1 through 105 as the boxes. We will use the elements in $A$ as the objects. From \eqref{6} we see that there are more objects than boxes and so there most exist two integers $m$ and $n$ such that
\begin{equation}\label{7}
a_m = a_n+15.
\end{equation}
Subtracting $a_n$ from both sides of \eqref{7} we obtain
\begin{equation}\label{8}
a_m-a_n = 15.
\end{equation}
From \eqref{8} we see that from the $n^{th}$ day to the $m^{th}$ day Cameron learned 15 new words.

In conclusion, we have shown that there exists some span of consecutive days during which Cameron learns exactly 15 words.

\end{proof}


\end{document}