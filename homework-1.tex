\documentclass[12pt]{article}
%\usepackage{geometry}
\usepackage[a4paper,centering,includehead]{geometry}
%\usepackage[paperwidth=6in, paperheight=9in]{geometry} %For Book Printing
%==============================
\usepackage{microtype}
%==============================
%Math Related Packages
\usepackage{amsmath, amsfonts, amssymb, amsthm, xfrac, mathtools}
\newtheorem{thm}{Theorem}
\begin{document}
\begin{flushright}
Russ Johnson\\
Homework 1\\
\today\\
\end{flushright}
Let $\sim$ be a binary relation on $\mathbb{Z}$ defined: for all integers $a$ and $b$, $a\sim b$ if and only if $2a+3b$ is a multiple of 5. We will show that the relation $\sim$ on $\mathbb{Z}$ is an equivalence relation. This means that the binary relation $\sim$ on $\mathbb{Z}$ is reflexive, symmetric, and transitive. We will prove that $\sim$ satisfies these properties on $\mathbb{Z}$ in this order.

\begin{thm}
The binary relation $\sim$ on $\mathbb{Z}$ satisfies the reflexive property.
\end{thm}
\begin{proof}
Let $a$ be an arbitrary integer. We will prove that $a\sim a$. The set of integers is a ring and so from the Distributive Property of Multiplication over Addition we obtain
\begin{align*}
2a+3a &= (2+3)a \\
&= 5a.
\end{align*}
We know that $5a$ is an integer from the closure property of multiplication on the set integers. From the definition of multiples, the integer $5a$ is a multiple of 5. 

We have therefore shown that for an arbitrary integer $a$, $a\sim a$, which means that the binary relation $\sim$ on $\mathbb{Z}$ is reflexive.
\end{proof}

\begin{thm}
The binary relation $\sim$ on $\mathbb{Z}$ satisfies the symmetric property.
\end{thm}
\begin{proof}
Let $a$ and $b$ be arbitrary integers such that $a\sim b$. From the definition of $a\sim b$ there exists some integer $k$ such that
\begin{equation}\label{1}
2a + 3b = 5k.
\end{equation}
Subtracting $5b$ from both sides of \eqref{1} we obtain
\begin{equation}\label{2}
2a - 2b = 5k - 5b.
\end{equation}
Applying the Distributive Property of Multiplication over Subtraction to \eqref{2} we obtain
\begin{equation}\label{3}
2(a-b) = 5(k-b).
\end{equation}
From \eqref{3} we see that $5\mid 2(a-b)$. Now, 5 is a prime number and 2 and $a-b$ are both integers from the closure properties of the set of all integers. We also know that if a prime divides the product of two integers, it must divide the first integer, the second integer, or both. We know that $5 \nmid 2$ and so it must be the case that $5\mid (a-b)$. From the definition of divides there must exist some integer $m$ such that
\begin{equation}\label{4}
a-b=5m.
\end{equation}
Adding the equations \eqref{1} and \eqref{4} together we obtain
\begin{equation}\label{5}
3a + 2b = 5k + 5m.
\end{equation}
Applying the Distributive Property of Multiplication over Addition to \eqref{5} we obtain
\begin{equation}\label{6}
3a + 2b = 5(k+m).
\end{equation}
Because addition is commutative in a ring (the set of all integers is a ring), from \eqref{6} we see that
\begin{equation}\label{7}
2b + 3a = 5(k+m).
\end{equation}
The set of integers is closed under addition and so $k+m$ is an integer. From this fact and \eqref{7} we see that $2b+3a$ is a multiple of 5. 

We have therefore shown that for the arbitrary integers $a$ and $b$ if $a\sim b$, then $b \sim a$. This means that the relation $\sim$ on $\mathbb{Z}$ is symmetric.
\end{proof}

\begin{thm}
The binary relation $\sim$ on $\mathbb{Z}$ satisfies the transitive property.
\end{thm}
\begin{proof}
Let $a$, $b$, and $c$ be arbitrary integers such that $a\sim b$ and $b\sim c$. We will prove that $a\sim c$. We know from $a\sim b$ and $b\sim c$ that there exist some integers $m$ and $n$
\begin{equation}\label{b1}
2a + 3b = 5m
\end{equation}
and
\begin{equation}\label{b2}
2b + 3c = 5n.
\end{equation}
Adding equations \eqref{b1} and \eqref{b2} together we obtain
\begin{equation}\label{b3}
2a + 5b +3c = 5m + 5n.
\end{equation}
Applying the Commutative Property of Addition to \eqref{b3} we obtain
\begin{equation}\label{b4}
2a + 3c + 5b = 5m + 5n.
\end{equation}
Subtracting $5b$ on \eqref{b4} we obtain
\begin{equation}\label{b5}
2a + 3c = 5m + 5n - 5b.
\end{equation}
Applying the distributive property of multiplication on \eqref{b5},
\begin{equation}\label{b6}
2a + 3c = 5(m+n-b).
\end{equation}
The set of integers is closed under addition and so $m+n-b$ is an integer. From this fact, \eqref{b6}, and the definition of multiples $2a + 3c$ is a multiple of 5 and so $a\sim c$.

In conclusion, we have shown that for the arbitrary integers $a$, $b$, and $c$ if $a\sim b$ and $b\sim c$, then $a\sim c$. This means that the binary relation $\sim$ on $\mathbb{Z}$ is transitive.
\end{proof}

Finally, from theorems 1, 2, and 3 we have shown that the binary relation $\sim$ is reflexive, symmetric, and transitive on $\mathbb{Z}$. Therefore, $\sim$ is an equivalence relation on $\mathbb{Z}$.

\newpage
There are five equivalences classes of the binary relation $\sim$ on $\mathbb{Z}$.
\begin{align*}
\overline{0} &= \{5x: x \in \mathbb{Z}\}\\
\overline{1} &= \{5x + 1: x \in \mathbb{Z}\}\\
\overline{2} &= \{5x + 2: x \in \mathbb{Z}\}\\
\overline{3} &= \{5x + 3: x \in \mathbb{Z}\}\\
\overline{4} &= \{5x + 4: x \in \mathbb{Z}\}\\
\end{align*}
Any other equivalence class is equal to one of these 5. For example $\overline{5} = \overline{0}$. This is do to the fact that the binary relation $\sim$ on $\mathbb{Z}$ is symmetric. The integer 0 is related to 5 and so 5 is related to 0.

Another way of looking at this is that when we talk about the equivalence class $\overline{a}$, $a$ is referred to as the representative of the equivalence class. In truth, any element in the set $\overline{a}$ can be used as a representative for that equivalence class. This means that for all elements $b \in \overline{a}$, $\overline{a}=\overline{b}$. Because $\overline{0}\cup\overline{1}\cup\overline{2}\cup\overline{3}\cup\overline{4} = \mathbb{Z}$, there are no other equivalence classes. That is, there does not exist an integer that is not in the set of all integers. Also, for any two of these equivalence classes, their intersection is equal to the null set. This is because if their were a common element they would necessarily be the same class.

\newpage

\noindent 2. a) Without this change in the rules the relation is not transitive. One counter example to transitivity is that $\text{MTH } 201\sim \text{MTH } 202$ and $\text{MTH } 202 \sim \text{MTH } 329$ but $\text{MTH } 201 \nsim \text{MTH } 329$.\\

\noindent c) MTH 097, MTH 180, MTH 386, MTH 387, MTH 399 are minimal elements of $(S,\preccurlyeq)$. Let these minimal elements be in the set M. The following is true. For each element $a\in M$, if $c \in S$ and $c \preccurlyeq a$, then $c = a$.  This implies that each of these elements is a minimal element.\\

\noindent d) MTH 496, MTH 465, MTH 409, MTH 402, MTH 441, MTH 410, MTH 323, MTH 345, MTH 431, MTH 408, MTH 327, MTH 360, MTH 229, and MTH 329 are all maximal elements of $(S, \preccurlyeq)$. Let these maximal elements be in the set M. The following is true. For each element $a\in M$, if $c \in S$ and $a \preccurlyeq c$, then $c = a$.  This implies that each of these elements is a maximal element.\\

\noindent e) There is a maximum element in $(S,\preccurlyeq)$ if and only if there exists exactly one maximal element in $(S,\preccurlyeq)$. This is not true for $(S,\preccurlyeq)$ and so there is no maximum element.\\

\noindent f) There is a minimum element in $(S,\preccurlyeq)$ if and only if there exists exactly one minimal element in $(S,\preccurlyeq)$. This is not true for $(S,\preccurlyeq)$ and so there is no minimum element.\\


For some background information I have been reading {\it Contemporary Abstract Algebra} by Gallian and {\it Discrete Mathematics and its Applications} by Rosen. Aside from this, I did not use any other outside source.

\end{document}