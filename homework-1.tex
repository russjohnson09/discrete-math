\documentclass[12pt]{article}
%\usepackage{geometry}
\usepackage[a4paper,centering,includehead]{geometry}
%\usepackage[paperwidth=6in, paperheight=9in]{geometry} %For Book Printing
%==============================
%\setlength{\parskip}{14pt plus 1pt minus 2pt}
\usepackage{microtype}
%==============================
%Math Related Packages
\usepackage{amsmath, amsfonts, amssymb, amsthm, xfrac, mathtools}
%==============================
\usepackage{framed}
%==============================
\usepackage{pdflscape}
%==============================
% For more complex numbering
%\usepackage{enumitem}
\usepackage{enumerate}
\usepackage{mdwlist}
%==============================
%For Importing inkscape svg files with latex font.
\usepackage{import}
%==============================
%Table Related
\usepackage{multirow, multicol, tabularx, longtable}
\usepackage[tableposition=below]{caption}
\captionsetup[longtable]{skip=1em}
%==============================
%Colors and Graphics
\usepackage{color}
\usepackage{graphicx}
%Hyperref
\usepackage[unicode=true, pdfusetitle,
 bookmarks=true,bookmarksnumbered=false,bookmarksopen=false,
 breaklinks=false,pdfborder={0 0 0},backref=false,colorlinks=false]
 {hyperref}
\usepackage{cite}
\usepackage{tocloft}
\usepackage{tikz}
%==============================
\theoremstyle{definition}
\newtheorem{thm}{Theorem}
\newtheorem{lem}{Lemma}
\theoremstyle{definition}
\newtheorem{ques}{Question}
\theoremstyle{definition}
\newtheorem{df}{Definition}
\theoremstyle{definition}
\newtheorem{dfn}{Definition}
\theoremstyle{definition}
\newtheorem{indfn}{Informal Definition}
%==============================
%\usepackage{parskip}
\begin{document}
\begin{flushright}
Russ Johnson\\
Homework 1\\
\today\\
\end{flushright}
Define the binary relation $\sim$ on $\mathbb{Z}$ by: for all integers $a$ and $b$, $a\sim b$ if and only if $2a+3b$ is a multiple of 5. We will show that the relation $\sim$ on $\mathbb{Z}$ is an equivalence relation. This means that the binary realtion $~$ on $\mathbb{Z}$ is reflexive, symmetric, and transitive. We will prove that $~$ satisfies these properties on $\mathbb{Z}$ in this order.

\begin{proof}
Let $a$ be an arbitrary integer. The set of integers is a ring and so from the distributive property of multiplication over addition,
\begin{align*}
2a+3a &= (2+3)a \\
&= 5a.
\end{align*}
\end{proof}

\begin{proof}
Let $a$ and $b$ be arbitrary integers where there exists an integer $k$ such that
\begin{equation}\label{1}
2a + 3b = 5k.
\end{equation}
Subtracting $5b$ from both sides of \eqref{1},
\begin{equation}
2a - 2b = 5k - 5b.
\end{equation}
\end{proof}

We know that $5a$ is an integer from the closure property of multiplication on the set integers. From the definition of multiples, the integer $5a$ is a multiple of 5. We have therefore shown that for an arbitrary integer $a$, $a\sim a$. This implies that the binary relatiion

For some background information I have been reading {\it Contemporary Abstract Algebra} by Gallian.  Aside from this, I did not use any other source.

\end{document}