\documentclass[12pt]{article}
%\usepackage{geometry}
\usepackage[a4paper,centering,includehead]{geometry}
%\usepackage[paperwidth=6in, paperheight=9in]{geometry} %For Book Printing
%==============================
\usepackage{microtype}
%==============================
%Math Related Packages
\usepackage{amsmath, amsfonts, amssymb, amsthm, xfrac, mathtools}
\begin{document}
\begin{flushright}
Russ Johnson\\
Homework 1\\
\today\\
\end{flushright}
Define the binary relation $\sim$ on $\mathbb{Z}$ by: for all integers $a$ and $b$, $a\sim b$ if and only if $2a+3b$ is a multiple of 5. We will show that the relation $\sim$ on $\mathbb{Z}$ is an equivalence relation. This means that the binary realtion $\sim$ on $\mathbb{Z}$ is reflexive, symmetric, and transitive. We will prove that $~$ satisfies these properties on $\mathbb{Z}$ in this order.

\begin{proof}
Let $a$ be an arbitrary integer. The set of integers is a ring and so from the distributive property of multiplication over addition,
\begin{align*}
2a+3a &= (2+3)a \\
&= 5a.
\end{align*}
We know that $5a$ is an integer from the closure property of multiplication on the set integers. From the definition of multiples, the integer $5a$ is a multiple of 5. We have therefore shown that for an arbitrary integer $a$, $a\sim a$. This implies that the binary relation $\sim$ on $\mathbb{Z}$ is reflexive.
\end{proof}

\begin{proof}
Let $a$ and $b$ be arbitrary integers where there exists an integer $k$ such that
\begin{equation}\label{1}
2a + 3b = 5k.
\end{equation}
Subtracting $5b$ from both sides of \eqref{1},
\begin{equation}\label{2}
2a - 2b = 5k - 5b.
\end{equation}
Applying the distributive property of multiplication to \eqref{2},
\begin{equation}\label{3}
2(a-b) = 5(k-b).
\end{equation}
Now $5\nmid 2$, but $5\mid 2(a-b)$. From the Fundamental Theorem of Arithmetic and \eqref{3} we know that there exists an integer $m$ such that
\begin{equation}\label{4}
a-b=5m
\end{equation}
Adding the equations \eqref{1} and \eqref{4} together,
\begin{equation}\label{5}
3a + 2b = 5k + 5m.
\end{equation}
Applying the distributive property of multiplication to \eqref{5},
\begin{equation}\label{6}
3a + 2b = 5(k+m)
\end{equation}
Because addition is commutative in a ring, from \eqref{6} we see that
\begin{equation}\label{7}
2b + 3a = 5(k+m).
\end{equation}
The set integers are closed under addition and so from \eqref{7} we see that $2b+3a$ is a multiple of 5. We have therefore shown that for the arbitrary integers $a$ and $b$ if $a\sim b$, then $b \sim a$. This implies that the relation $\sim$ on $\mathbb{Z}$ is symmetric.
\end{proof}



For some background information I have been reading {\it Contemporary Abstract Algebra} by Gallian and   Aside from this, I did not use any other source.

\end{document}