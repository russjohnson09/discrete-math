\documentclass[12pt]{article}
%\usepackage{geometry}
\usepackage[a4paper,centering,includehead]{geometry}
%\usepackage[paperwidth=6in, paperheight=9in]{geometry} %For Book Printing
%==============================
\usepackage{microtype}
%==============================
%Math Related Packages
\usepackage{amsmath, amsfonts, amssymb, amsthm, xfrac, mathtools}
\begin{document}
\begin{flushright}
Russ Johnson\\
Homework 1\\
\today\\
\end{flushright}
Define the binary relation $\sim$ on $\mathbb{Z}$ by: for all integers $a$ and $b$, $a\sim b$ if and only if $2a+3b$ is a multiple of 5. We will show that the relation $\sim$ on $\mathbb{Z}$ is an equivalence relation. This means that the binary realtion $\sim$ on $\mathbb{Z}$ is reflexive, symmetric, and transitive. We will prove that $\sim$ satisfies these properties on $\mathbb{Z}$ in this order.

\begin{proof}
Let $a$ be an arbitrary integer. The set of integers is a ring and so from the distributive property of multiplication over addition,
\begin{align*}
2a+3a &= (2+3)a \\
&= 5a.
\end{align*}
We know that $5a$ is an integer from the closure property of multiplication on the set integers. From the definition of multiples, the integer $5a$ is a multiple of 5. We have therefore shown that for an arbitrary integer $a$, $a\sim a$. This implies that the binary relation $\sim$ on $\mathbb{Z}$ is reflexive.
\end{proof}

\begin{proof}
Let $a$ and $b$ be arbitrary integers where there exists an integer $k$ such that
\begin{equation}\label{1}
2a + 3b = 5k.
\end{equation}
Subtracting $5b$ from both sides of \eqref{1},
\begin{equation}\label{2}
2a - 2b = 5k - 5b.
\end{equation}
Applying the distributive property of multiplication to \eqref{2},
\begin{equation}\label{3}
2(a-b) = 5(k-b).
\end{equation}
From \eqref{3} we see that $5\mid 2(a-b)$ and from this and the Fundemental Theorem of Arithmetic we know that $5\mid 2$ or $5\mid (a-b)$. Since $5 \nmid 2$ we know that $5\mid (a-b)$ and therefore there must exist some integer $m$ such that
\begin{equation}\label{4}
a-b=5m.
\end{equation}
Adding the equations \eqref{1} and \eqref{4} together,
\begin{equation}\label{5}
3a + 2b = 5k + 5m.
\end{equation}
Applying the distributive property of multiplication to \eqref{5},
\begin{equation}\label{6}
3a + 2b = 5(k+m).
\end{equation}
Because addition is commutative in a ring, from \eqref{6} we see that
\begin{equation}\label{7}
2b + 3a = 5(k+m).
\end{equation}
The set of integers is closed under addition and so from \eqref{7} we see that $2b+3a$ is a multiple of 5. We have therefore shown that for the arbitrary integers $a$ and $b$ if $a\sim b$, then $b \sim a$. This implies that the relation $\sim$ on $\mathbb{Z}$ is symmetric.
\end{proof}

\begin{proof}
Let $a$, $b$, and $c$ be arbitrary integers where there exist integers $m$ and $n$ such that
\begin{equation}\label{b1}
2a + 3b = 5m
\end{equation}
and
\begin{equation}\label{b2}
2b + 3c = 5n.
\end{equation}
Adding equations \eqref{b1} and \eqref{b2} together,
\begin{equation}\label{b3}
2a + 5b +3c = 5m + 5n.
\end{equation}
Applying the commutative property of addition to \eqref{b3},
\begin{equation}\label{b4}
2a + 3c + 5b = 5m + 5n.
\end{equation}
Subtracting $5b$ on \eqref{b4},
\begin{equation}\label{b5}
2a + 3c = 5m + 5n - 5b.
\end{equation}
Applying the distributive property of multiplication on \eqref{b5},
\begin{equation}\label{b6}
2a + 3c = 5(m+n-b).
\end{equation}
Integers are closed under addition and so from \eqref{b6} and the definition of multiples $2a + 3c$ is a multiple of 5. We have therefore shown that for the arbitrary integers $a$, $b$, and $c$ if $a\sim b$ and $b\sim c$, then $a\sim c$. This implies that the binary relation $\sim$ on $\mathbb{Z}$ is transitive.
\end{proof}
We have shown that the binary relation $\sim$ is reflexive, symmetric, and transitive on $\mathbb{Z}$ and therefore $\sim$ is an equivalence relation on $\mathbb{Z}$.

There are five equivalences classes of the binary relation $\sim$ on $\mathbb{Z}$.
\begin{align*}
[0]_\sim &= \{5x: x \in \mathbb{Z}\}\\
[1]_\sim &= \{5x + 1: x \in \mathbb{Z}\}\\
[2]_\sim &= \{5x + 2: x \in \mathbb{Z}\}\\
[3]_\sim &= \{5x + 3: x \in \mathbb{Z}\}\\
[4]_\sim &= \{5x + 4: x \in \mathbb{Z}\}\\
\end{align*}
Any other equivalenece class is equal to one of these 5. For example $[6]_\sim = [1]_\sim$.\\

\noindent 2. a) Without this change in the rules the relation is not transitive. One counter example to transitivity is that $\text{MTH } 201\sim \text{MTH } 202$ and $\text{MTH } 202 \sim \text{MTH } 329$ but $\text{MTH } 201 \nsim \text{MTH } 329$.\\

\noindent c) MTH 097, MTH 180, MTH 386, MTH 387, MTH 399 are minimal elements of $(S,\preccurlyeq)$. Let these minimal elements be in the set M. The following is true. For each element $a\in M$, if $c \in S$ and $c \preccurlyeq a$, then $c = a$.  This implies that each of these elements is a minimal element.\\

\noindent d) MTH 496, MTH 465, MTH 409, MTH 402, MTH 441, MTH 410, MTH 323, MTH 345, MTH 431, MTH 408, MTH 327, MTH 360, MTH 229, and MTH 329 are all maximal elements of $(S, \preccurlyeq)$. Let these maximal elements be in the set M. The following is true. For each element $a\in M$, if $c \in S$ and $a \preccurlyeq c$, then $c = a$.  This implies that each of these elements is a maximal element.\\

\noindent e) There is a maximum element in $(S,\preccurlyeq)$ if and only if there exists exactly one maximal element in $(S,\preccurlyeq)$. This is not true for $(S,\preccurlyeq)$ and so there is no maximum element.\\

\noindent f) There is a minimum element in $(S,\preccurlyeq)$ if and only if there exists exactly one minimal element in $(S,\preccurlyeq)$. This is not true for $(S,\preccurlyeq)$ and so there is no minimum element.\\


For some background information I have been reading {\it Contemporary Abstract Algebra} by Gallian and {\it Discrete Mathematics and its Applications} by Rosen. Aside from this, I did not use any other outside source.

\end{document}