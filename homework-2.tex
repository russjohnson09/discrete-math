\documentclass[12pt]{article}
%\usepackage{geometry}
\usepackage[a4paper,centering,includehead]{geometry}
%\usepackage[paperwidth=6in, paperheight=9in]{geometry} %For Book Printing
\usepackage{changepage}
\usepackage{color}
\usepackage[usenames,dvipsnames,svgnames,table]{xcolor}
%==============================
\usepackage{microtype}
%==============================
%Math Related Packages
\usepackage{amsmath, amsfonts, amssymb, amsthm, xfrac, mathtools}
\begin{document}
\begin{flushright}
Russ Johnson\\
Homework 3\\
\today\\
\end{flushright}

We will be looking at binomials and their relation with Pascal's triangle.
First, we will show that 
\[\sum\limits_{k=2}^6\binom{k}{2} = \binom{7}{3}.\]
To show this is true we first find the value of the following summation $\sum\limits_{k=2}^6\binom{k}{2}.$
\begin{equation}\label{1}
\sum\limits_{k=2}^6\binom{k}{2} = \binom{2}{2} + \binom{3}{2} + \binom{4}{2} + \binom{5}{2} + \binom{6}{2} = 1+3+6+10+15 = 35.
\end{equation} \\
We then find the value of $\binom{7}{3}.$
\begin{equation}\label{2}
\binom{7}{3} = 35.
\end{equation}
From \eqref{1} and \eqref{2} we see that
\[\sum\limits_{k=2}^6\binom{k}{2} = \binom{7}{3}.\]

Another way to look at this problem is with pascals triangle.
\begin{adjustwidth}{-2cm}{}
{\color{gray!50}
\begin{tabular}{rccccccccccccccccccccc}
   &    &    &    &    &    &    &    &    &    &  1\\\noalign{\smallskip\smallskip}
   &    &    &    &    &    &    &    &    &  1 &    &  1\\\noalign{\smallskip\smallskip}
   &    &    &    &    &    &    &    &  1 &    &  2 &    &  {\color{black}1}\\\noalign{\smallskip\smallskip}
   &    &    &    &    &    &    &  1 &    &  3 &    &  {\color{black}3} &    &  1\\\noalign{\smallskip\smallskip}
   &    &    &    &    &    &  1 &    &  4 &    &  {\color{black}6} &    &  4 &    &  1\\\noalign{\smallskip\smallskip}
   &    &    &    &    &  1 &    &  5 &    & {\color{black}10} &    & 10 &    &  5 &    &  1\\\noalign{\smallskip\smallskip}
   &    &    &    &  1 &    &  6 &    & {\color{black}15} &    & 20 &    & 15 &    &  6 &    &  1\\\noalign{\smallskip\smallskip}
   &    &    &  1 &    &  7 &    & 21 &    & {\color{black}35} &    & 35 &    & 21 &    &  7 &    &  1\\\noalign{\smallskip\smallskip}
   &    &  1 &    &  8 &    & 28 &    & 56 &    & 70 &    & 56 &    & 28 &    &  8 &    &  1\\\noalign{\smallskip\smallskip}
   &  1 &    &  9 &    & 36 &    & 84 &    & 126 &    & 126 &    & 84 &    & 36 &    &  9 &    &  1\\\noalign{\smallskip\smallskip}
 1 &    & 10 &    & 45 &    & 120 &    & 210 &    & 252 &    & 210 &    & 120 &    & 45 &    & 10 &    &  1\\\noalign{\smallskip\smallskip}
\end{tabular}
}

\end{adjustwidth}
From this we could say that the sum of the elements along the diagonal starting at $\binom{2}{2}$ and heading south-west is equal to the element south-east of the last element added.

Now we will construct a table to help us look for some general result to use as a conjecture.

\begingroup
\renewcommand*{\arraystretch}{1.5}
\begin{center}
$
\begin{array}{|c|c|c|c|}
\hline
r&n&\sum\limits_{k=r}^n\binom{k}{r}\\\hline
1&1&\binom{2}{2}=1\\\hline
1&2&\binom{3}{2}=3\\\hline
2&2&\binom{3}{3}=1\\\hline
2&3&\binom{4}{3}=4\\\hline
2&4&\binom{5}{3}=10\\\hline
\end{array}
$
\end{center}
\endgroup

From this table we can form the following conjecture. For all positive integers $r$ and $n$ such that $r\leq n$,
\[\sum\limits_{k=r}^{n}\binom{k}{r} = \binom{n+1}{r+1}.\]
Next we will prove this conjecture.

\begin{proof}
We will prove our conjecture using the inductive method. For our base case we assume that $n = r.$ Note that we will be proving that our conjecture is true for $n \geq r$ and so it is okay for our base case to start at $r$. Now we can clearly see that
\begin{align*}
\sum\limits_{k=r}^r\binom{k}{r} &= \binom{r}{r}\\
&=1\\
&=\binom{r+1}{r+1}.\\
\end{align*}
Now to complete the inductive proof we must show that for some $n\in \mathbb{N}$ with $n\geq r$
\[\sum\limits_{k=r}^n\binom{k}{r} = \binom{n+1}{r+1}\]
implies
\[\sum\limits_{k=r}^{n+1}\binom{k}{r} = \binom{n+2}{r+1}.\]
We see that
\begin{align*}
\sum\limits_{k=r}^{n+1}\binom{k}{r} &= \sum\limits_{k=r}^n\binom{k}{r}+\binom{n+1}{r}\\
&= \binom{n+1}{r+1} + \binom{n+1}{r}\\
\end{align*}
and from Pascal's rule we now that
\[\binom{n+1}{r+1} + \binom{n+1}{r} = \binom{n+2}{r+1}.\]
From this we completed our proof by induction and have shown that for all positive integers $r$ and $n$ such that $r\leq n$,
\[\sum\limits_{k=r}^{n}\binom{k}{r} = \binom{n+1}{r+1}.\]
\end{proof}

For the next problem we will show that there was a miscount of school children.
\begin{proof}
Let $U$ be the set of all the schoolchildren that walked into the store, let $P$ be the set of children that bought popsicles, $G$ the set that bought gum, and $B$ the set that bought candy bars.

\noindent We know the following is true.
\[
\begin{array}{cccc}
|U| = 15		&	|P| = 10		&	|G| = 7			&	|B| = 12	\\
|P\cap G| = 5	&	|P\cap B| = 6	&	|G\cap B| = 2	&				\\
\end{array}
\]

We also now that
\[|P\cup G\cup C| \leq |U|,\]
\[0\leq |P\cap G\cap B|,\]
and
\[|P|+|G|+|B|-(|P\cap G|+|P\cap B|+|G\cap B|)+|P\cap G\cap B| = |P\cup G\cup C|\]
From this we obtain the following
\begin{align*}
|P|+|G|+|B|-(|P\cap G|+|P\cap B|+|G\cap B|)+|P\cap G\cap B| &\leq |U|\\
10+7+12-(5+6+2)+|P\cap G\cap B|&\leq 15\\
16+|P\cap G\cap B|&\leq 15\\
|P\cap G\cap B|&\leq -1
\end{align*}
The intersection $|P\cap G\cap B|$ being anything less than zero is a contradiction and so there was a miscount of schoolchildren.
\end{proof}

\end{document}